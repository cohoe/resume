%%%%%%%%%%%%%%%%%%%%%%%%%%%%%%%%%%%%%%%%%%%%%%%%%%%%%%%%%%%%%%%%%%%%%%%%
%%%%%%%%%%%%%%%%%%%%%% Simple LaTeX CV Template %%%%%%%%%%%%%%%%%%%%%%%%
%%%%%%%%%%%%%%%%%%%%%%%%%%%%%%%%%%%%%%%%%%%%%%%%%%%%%%%%%%%%%%%%%%%%%%%%

%%%%%%%%%%%%%%%%%%%%%%%%%%%%%%%%%%%%%%%%%%%%%%%%%%%%%%%%%%%%%%%%%%%%%%%%
%% NOTE: If you find that it says                                     %%
%%                                                                    %%
%%                           1 of ??                                  %%
%%                                                                    %%
%% at the bottom of your first page, this means that the AUX file     %%
%% was not available when you ran LaTeX on this source. Simply RERUN  %%
%% LaTeX to get the ``??'' replaced with the number of the last page  %%
%% of the document. The AUX file will be generated on the first run   %%
%% of LaTeX and used on the second run to fill in all of the          %%
%% references.                                                        %%
%%%%%%%%%%%%%%%%%%%%%%%%%%%%%%%%%%%%%%%%%%%%%%%%%%%%%%%%%%%%%%%%%%%%%%%%

%%%%%%%%%%%%%%%%%%%%%%%%%%%% Document Setup %%%%%%%%%%%%%%%%%%%%%%%%%%%%

% Don't like 10pt? Try 11pt or 12pt
\documentclass[10pt]{article}

% The automated optical recognition software used to digitize resume
% information works best with fonts that do not have serifs. This
% command uses a sans serif font throughout. Uncomment both lines (or at
% least the second) to restore a Roman font (i.e., a font with serifs).
%%\usepackage{times}
%%\renewcommand{\familydefault}{\sfdefault}

% The OCR software also has a hard time with italics. These commands get
% rid of the two common ways to italicize text in LaTeX. Get rid of them
% to turn italics back on.
\renewcommand\emph[1]{#1}
\renewcommand\textit[1]{\underline{#1}}

% This is a helpful package that puts math inside length specifications
\usepackage{calc}

% Layout: Puts the section titles on left side of page
\reversemarginpar

%
%         PAPER SIZE, PAGE NUMBER, AND DOCUMENT LAYOUT NOTES:
%
% The next \usepackage line changes the layout for CV style section
% headings as marginal notes. It also sets up the paper size as either
% letter or A4. By default, letter was used. If A4 paper is desired,
% comment out the letterpaper lines and uncomment the a4paper lines.
%
% As you can see, the margin widths and section title widths can be
% easily adjusted.
%
% ALSO: Notice that the includefoot option can be commented OUT in order
% to put the PAGE NUMBER *IN* the bottom margin. This will make the
% effective text area larger.
%
% IF YOU WISH TO REMOVE THE ``of LASTPAGE'' next to each page number,
% see the note about the +LP and -LP lines below. Comment out the +LP
% and uncomment the -LP.
%
% IF YOU WISH TO REMOVE PAGE NUMBERS, be sure that the includefoot line
% is uncommented and ALSO uncomment the \pagestyle{empty} a few lines
% below.
%

%% Use these lines for letter-sized paper
\usepackage[paper=letterpaper,
            includefoot, % Uncomment to put page number above margin
            marginparwidth=1in,     % Length of section titles
            marginparsep=.05in,       % Space between titles and text
            margin=.4in,               % 1 inch margins
            includemp]{geometry}

%% Use these lines for A4-sized paper
%\usepackage[paper=a4paper,
%            %includefoot, % Uncomment to put page number above margin
%            marginparwidth=30.5mm,    % Length of section titles
%            marginparsep=1.5mm,       % Space between titles and text
%            margin=25mm,              % 25mm margins
%            includemp]{geometry}

%% More layout: Get rid of indenting throughout entire document
\setlength{\parindent}{0in}

\usepackage[shortlabels]{enumitem}


%%% Setup header and footer (with page number and possible last page)
%
% The first block sets up pages 2--end
% The second block sets up page 1 formatting
%
%%%
%
% NOTE: comment the +LP lines and uncomment the -LP lines to have page
%       numbers without the ``of ##'' last page reference)
%
% NOTE: uncomment the \pagestyle{empty} line to get rid of all page
%       numbers on pages 2--end. To get rid of page numbers on page 1,
%       comment out the \thispagestyle{plain} line on the first page
%       below.
%       (also make sure includefoot is commented out above)
%
\usepackage{lastpage}
\usepackage{fancyhdr}
\pagestyle{fancy}
%\pagestyle{empty}      % Uncomment this to get rid of page numbers
\fancyhf{}\renewcommand{\headrulewidth}{0pt}
\fancyfootoffset{\marginparsep+\marginparwidth}
\newlength{\footpageshift}
\setlength{\footpageshift}
          {0.5\textwidth+0.5\marginparsep+0.5\marginparwidth-2in}

%%%% PAGES 2--9 NUMBERING:
%% These two lines put page number in upper-right corner of pages 2--end
%%\rhead{Pavlic, p.~\arabic{page} of \protect\pageref*{LastPage}}   % +LP
%%\rhead{Pavlic, p.~\arabic{page}}                                 % -LP

%% These lines put page number in bottom (center) of pages 2--end
%\lfoot{\hspace{\footpageshift}%
%       \parbox{4in}{\, \hfill %
%                    \arabic{page} of \protect\pageref*{LastPage} % +LP
%%                    \arabic{page}                               % -LP
%                    \hfill \,}}
%%%% END PAGE 2--9 NUMBERING

%%%% PAGE 1 NUMBERING:
\makeatletter
\let\oldps@plain\ps@plain
\renewcommand{\ps@plain}{\oldps@plain%
\renewcommand{\@evenfoot}{\hspace*{-\footpageshift}\hfil %
    p.~\arabic{page} of \protect\pageref*{LastPage} % +LP
%    p.~\arabic{page}                               % -LP
    \hfil}%
\renewcommand{\@oddfoot}{\@evenfoot}}
\makeatother
%%%% END PAGE 1 NUMBERING

% Finally, give us PDF bookmarks and colored links
%
% NOTE: Some OCR software might be negatively affected by hyperlinks. So
%       most employers recommend the draft option here. Alternatively,
%       making all links black (as opposed to darkblue) should hopefully
%       prevent problems with most OCR.
%
% (to enable hyperlinks and bookmarks, comment out ``draft'' line;
%  to disable hyperlinks and bookmarks, uncomment ``draft'' line)
\usepackage{color,hyperref}
\definecolor{darkblue}{rgb}{0.0,0.0,0.3}
\hypersetup{breaklinks,colorlinks,
            linkcolor=black,urlcolor=black,
            anchorcolor=black,citecolor=black,
            %linkcolor=darkblue,urlcolor=darkblue,
            %anchorcolor=darkblue,citecolor=darkblue,
            %draft
            }

%%%%%%%%%%%%%%%%%%%%%%%% End Document Setup %%%%%%%%%%%%%%%%%%%%%%%%%%%%


%%%%%%%%%%%%%%%%%%%%%%%%%%% Helper Commands %%%%%%%%%%%%%%%%%%%%%%%%%%%%

%%% HEADING AT TOP OF CURRICULUM VITAE

% The title (name) with a horizontal rule under it
% (optional argument typesets an object right-justified across from name
%  as well)
%
% Usage: \makeheading{name}
%        OR
%        \makeheading[right_object]{name}
%
% Place at top of document. It should be the first thing.
% If ``right_object'' is provided in the square-braced optional
% argument, it will be right justified on the same line as ``name'' at
% the top of the CV. For example:
%
%       \makeheading[\emph{Curriculum vitae}]{Your Name}
%
% will put an emphasized ``Curriculum vitae'' at the top of the document
% as a title. Likewise, a picture could be included:
%
%   \makeheading[\includegraphics[height=1.5in]{my_picutre}]{Your Name}
%
% the picture will be flush right across from the name.
\newcommand{\makeheading}[2][]%
        {\hspace*{-\marginparsep minus \marginparwidth}%
        \begin{minipage}[t]{\textwidth+\marginparwidth+\marginparsep}%
        	{\LARGE \bfseries #2 \hfill #1}\\[-0.15\baselineskip]%
                \rule{\columnwidth}{1pt}%
		\halfblankline
        \end{minipage}}

%%% SECTION HEADINGS

% The section headings. Flush left in small caps down pseudo-margin.
%
% Usage: \section{section name}
\renewcommand{\section}[1]{\pagebreak[3]%
    \vspace{1\baselineskip}%
    \phantomsection\addcontentsline{toc}{section}{#1}%
    \noindent\llap{\scshape\smash{\parbox[t]{\marginparwidth}{\hyphenpenalty=10000\raggedright #1}}}%
    \vspace{-\baselineskip}\par}

%%% LISTS

% This macro alters a list by removing some of the space that follows the list
% (is used by lists below)
\newcommand*\fixendlist[1]{%
    \expandafter\let\csname preFixEndListend#1\expandafter\endcsname\csname end#1\endcsname
    \expandafter\def\csname end#1\endcsname{\csname preFixEndListend#1\endcsname\vspace{-0.6\baselineskip}}}

% These macros help ensure that items in outer-type lists do not get
% separated from the next line by a page break
% (they are used by lists below)
\let\originalItem\item
\newcommand*\fixouterlist[1]{%
    \expandafter\let\csname preFixOuterList#1\expandafter\endcsname\csname #1\endcsname
    \expandafter\def\csname #1\endcsname{\csname preFixOuterList#1\endcsname\let\oldItem\item\def\item{\pagebreak[2]\oldItem}}
    \expandafter\let\csname preFixOuterListend#1\expandafter\endcsname\csname end#1\endcsname
    \expandafter\def\csname end#1\endcsname{\let\item\oldItem\csname preFixOuterListend#1\endcsname}}
\newcommand*\fixinnerlist[1]{%
    \expandafter\let\csname preFixInnerList#1\expandafter\endcsname\csname #1\endcsname
    \expandafter\def\csname #1\endcsname{\let\oldItem\item\let\item\originalItem\csname preFixInnerList#1\endcsname}
    \expandafter\let\csname preFixInnerListend#1\expandafter\endcsname\csname end#1\endcsname
    \expandafter\def\csname end#1\endcsname{\csname preFixInnerListend#1\endcsname\let\item\oldItem}}

% An itemize-style list with lots of space between items
%
% Usage:
%   \begin{outerlist}
%       \item ...    % (or \item[] for no bullet)
%   \end{outerlist}
\newlist{outerlist}{itemize}{3}
    \setlist[outerlist]{label=\enskip\textbullet,leftmargin=*}
    \fixendlist{outerlist}
    \fixouterlist{outerlist}

% An environment IDENTICAL to outerlist that has better pre-list spacing
% when used as the first thing in a \section
%
% Usage:
%   \begin{lonelist}
%       \item ...    % (or \item[] for no bullet)
%   \end{lonelist}
\newlist{lonelist}{itemize}{3}
    \setlist[lonelist]{label=\enskip\textbullet,leftmargin=*,partopsep=0pt,topsep=0pt}
    \fixendlist{lonelist}
    \fixouterlist{lonelist}

% An itemize-style list with little space between items
%
% Usage:
%   \begin{innerlist}
%       \item ...    % (or \item[] for no bullet)
%   \end{innerlist}
\newlist{innerlist}{itemize}{3}
    \setlist[innerlist]{label=\enskip\textbullet,leftmargin=*,parsep=0pt,itemsep=0pt,topsep=0pt,partopsep=0pt}
    \fixinnerlist{innerlist}

% An environment IDENTICAL to innerlist that has better pre-list spacing
% when used as the first thing in a \section
%
% Usage:
%   \begin{loneinnerlist}
%       \item ...    % (or \item[] for no bullet)
%   \end{loneinnerlist}
\newlist{loneinnerlist}{itemize}{3}
    \setlist[loneinnerlist]{label=\enskip\textbullet,leftmargin=*,parsep=0pt,itemsep=0pt,topsep=0pt,partopsep=0pt}
    \fixendlist{loneinnerlist}
    \fixinnerlist{loneinnerlist}

%%% EXTRA SPACE

% To add some paragraph space between lines.
% This also tells LaTeX to preferably break a page on one of these gaps
% if there is a needed pagebreak nearby.
\newcommand{\blankline}{\quad\pagebreak[3]}
\newcommand{\halfblankline}{\quad\vspace{-0.5\baselineskip}\pagebreak[3]}

%%% FORMATTING MACROS

% Uses hyperref to link DOI
\newcommand\doilink[1]{\href{http://dx.doi.org/#1}{#1}}
\newcommand\doi[1]{doi:\doilink{#1}}

% For \url{SOME_URL}, links SOME_URL to the url SOME_URL
\providecommand*\url[1]{\href{#1}{#1}}
% Same as above, but pretty-prints SOME_URL in teletype fixed-width font
\renewcommand*\url[1]{\href{#1}{\texttt{#1}}}

% For \email{ADDRESS}, links ADDRESS to the url mailto:ADDRESS
\providecommand*\email[1]{\href{mailto:#1}{#1}}
% Same as above, but pretty-prints ADDRESS in teletype fixed-width font
%\renewcommand*\email[1]{\href{mailto:#1}{\texttt{#1}}}

%\providecommand\BibTeX{{\rm B\kern-.05em{\sc i\kern-.025em b}\kern-.08em
%    T\kern-.1667em\lower.7ex\hbox{E}\kern-.125emX}}
%\providecommand\BibTeX{{\rm B\kern-.05em{\sc i\kern-.025em b}\kern-.08em
%    \TeX}}
\providecommand\BibTeX{{B\kern-.05em{\sc i\kern-.025em b}\kern-.08em
    \TeX}}
\providecommand\Matlab{\textsc{Matlab}}

%%%%%%%%%%%%%%%%%%%%%%%% End Helper Commands %%%%%%%%%%%%%%%%%%%%%%%%%%%

%%%%%%%%%%%%%%%%%%%%%%%%% Begin CV Document %%%%%%%%%%%%%%%%%%%%%%%%%%%%
\begin{document}
%% BEGIN FOOTER %%
\lfoot{Page {\thepage} of {\pageref{LastPage}}}
% A digital copy of this document is available at \href{https://resume.grantcohoe.com}{https://resume.grantcohoe.com}}

%% END FOOTER %%
\makeheading{Grant Cohoe}

%% BEGIN CONTACT INFO %%
\section{Contact Information}
\emph{E-mail:} \email{jobs@grantcohoe.com}\\
\emph{Web:} \href{https://www.grantcohoe.com} {https://www.grantcohoe.com}

\emph{Resume:} \href{https://resume.grantcohoe.com} {https://resume.grantcohoe.com}
%% END CONTACT INFO


%% BEGIN SUMMARY %%
\section{Summary}
Operations engineer with excellent problem solving abilities and extensive knowledge in emerging, current, and legacy technologies capable of making practice and policy recommendations across the enterprise while enabling others to do their best work.
%% END SUMMARY


%% BEGIN SKILLS %%
\section{Skills}

\textbf{Orchestration Tools}: Ansible, Terraform, Helm, Puppet

\textbf{Platforms \& Providers}: Microsoft Azure, Amazon Web Services, Kubernetes

\textbf{Services \& Applications}: Docker, Prometheus, Grafana, ElasticSearch, Consul, RabbitMQ, Vault

\textbf{Databases}: MySQL/MariaDB, PostgreSQL, Cassandra, DynamoDB, Redis, SQLite

\textbf{Network}: Cisco, Juniper, F5, Ubiquiti, VPN (OpenVPN, IPsec, Wireguard), Tr{\ae}fik, Istio

\textbf{Operating Systems}: Linux (Focus on RHEL/CentOS/Fedora \& Ubuntu), FreeBSD, Windows, MacOS

\textbf{Languages}: Python, Swift, NodeJS, Bash

\textbf{Build \& Deploy Tools}: Spacelift, TeamCity, Harness, Buildkite, Jenkins, Git
%% END SKILLS %%


%% BEGIN EXPERIENCE %%
\section{Professional Experience}

\href{https://www.klaviyo.com/}{\textbf{Klaviyo}},
Boston MA
\begin{outerlist}
	\item[] \textit{Senior Site Reliability Engineer} (Infrastructure Lifecycle \& Paved Path)
        \hfill \textbf{Feb 2024 to Present}
	\begin{innerlist}
		\item Maintain config management (Puppet, Terraform) and provisioning systems (cloud-init, Spacelift).
		\item Develop and maintain the internal Kubernetes platform and facilitate adoption.
		\item Leverage LLM ("AI") solutions to deliver a wide array of infrastructure projects.
        \end{innerlist}
\end{outerlist}
\halfblankline

\href{https://www.bnymellon.com/us/en/solutions/asset-managers/data-analytics.html}{\textbf{BNY Mellon | Eagle Investment Systems}},
Wellesley MA
\begin{outerlist}
	\item[] \textit{Principal Software Engineer} (DevOps)
        \hfill \textbf{Nov 2021 to Dec 2023}
	\begin{innerlist}
		\item Orchestrate Kubernetes deployments in a cloud native Microsoft Azure environment.
		\item Lead the exploration and proof-of-concept for service mesh implementation.
		\item Provide support and mentoring to internal teams and contractors to enable successful solution delivery.
        \end{innerlist}
\end{outerlist}
\halfblankline

\href{https://www.constantcontact.com/}{\textbf{Constant Contact}},
Waltham MA
\begin{outerlist}
	\item[] \textit{Principal System Engineer} (TechOps Systems)
        \hfill \textbf{Mar 2020 to Oct 2021}
	\begin{innerlist}
		\item Design system, platform, and architectural best practices and processes to fit dynamic business cases.
		\item Provide extensive consultation and troubleshooting to multiple brands across the enterprise.
		\item Build and test the on-premesis Kubernetes platform integrating with many existing systems.
        \end{innerlist}

	\item[] \textit{Senior System Engineer} (TechOps Systems)
        \hfill \textbf{Jan 2017 to Mar 2020}
	\begin{innerlist}
		\item Develop and maintain extensive automation infrastructure and system platforms.
		\item Oversee operation of all Cassandra clusters in production and pre-prod environments.
		\item Perform assessment of acquired businesses and document all aspects of their platform.
        \end{innerlist}
\end{outerlist}
\halfblankline

\href{https://www.rsa.com/}{\textbf{RSA The Security Division of EMC}},
Bedford MA
\begin{outerlist}
   	\item[] \textit{Senior System Administrator} (Engineering Lab Services / Release Engineering)
        \hfill \textbf{Jul 2014 to Jan 2017}
        \begin{innerlist}
		\item Responsible for driving implementation initiatives such as VPN tunneling \& system automation. 
		\item Work with cross-functional teams across the world to deliver projects in a complete and timely manner. 		
		\item Provide level 3 support to all systems and services in the global engineering infrastructure.
        \end{innerlist}

    	\item[] \textit{System Administrator} (Engineering Lab Services / Release Engineering)
        \hfill \textbf{Jul 2013 to Jul 2014}
        \begin{innerlist}
               \item Assist in the development \& operations of the global product engineering labs and release infrastructure.
               \item Develop infrastructure software and tools to provide visability into lab resources and utilization.
        \end{innerlist}
\end{outerlist}
\halfblankline

\href{https://www.campcomputer.com/}{\textbf{Camp Fitch YMCA}},
North Springfield PA
\begin{outerlist}
	\item[] \textit{Director} (Tech Focus Experience / Computer Camp)
	\hfill \textbf{Jun 2013 to Present}
	\begin{innerlist}
		\item Plan \& operate a two-week program teaching children introductory computer topics.
		\item Primary point of contact with parents and full-time camp staff. 
		\item Responsible for facilities, on-site operations, staffing, and mentoring of younger staff.
	\end{innerlist}
\end{outerlist}
%% END EXPERIENCE %%


\pagebreak
\makeheading{Grant Cohoe}


%% BEGIN PROJECTS %%
\section{Projects}

\textbf{Charter Cruise Social Media Platform} - \href{https://github.com/jocosocial/swiftarr}{https://github.com/jocosocial/swiftarr}
\begin{outerlist}
    \item[]
            \begin{innerlist}
	\item Contribute to a bespoke social media platform built for the unique culture of a nerd cruise.
	\item Develop core application features, deployment \& testing orchestration.
            \end{innerlist}
\end{outerlist}
\halfblankline

\textbf{Charter Cruise Android App} - \href{https://github.com/jocosocial/tricordarr}{https://github.com/jocosocial/tricordarr}
\begin{outerlist}
    \item[]
            \begin{innerlist}
	\item Built a React Native app for Android devices using the Twitarr service API (see above).
	\item Successfully implemented push notifications in a zero-internet environment.
            \end{innerlist}
\end{outerlist}
\halfblankline

\textbf{Internet Connection Monitor} - \href{https://github.com/CFCC/TurboStacks/tree/main/metrics}{https://github.com/CFCC/TurboStacks/tree/main/metrics}
\begin{outerlist}
    \item[]
            \begin{innerlist}
		\item Developed a series of dashboards to display connection metrics and determine perceived network quality.
                \item Utilized Docker Compose, Grafana, Prometheus, and the Blackbox Exporter.
            \end{innerlist}
\end{outerlist}
\halfblankline

\textbf{Cocktail Recipe Research Platform} - \href{https://github.com/cohoe/amari}{https://github.com/cohoe/amari}
\begin{outerlist}
    \item[]
            \begin{innerlist}
                \item Built a full-stack application platform to manage and query recipes in a variety of domain-specific ways.
                \item Stack consists of React, Flask (RESTx), Zookeeper, Redis, ElasticSearch, PostgreSQL, Cognito.
            \end{innerlist}
\end{outerlist}
\halfblankline

\textbf{Puppet Infrastructure for Computer Camp} - \href{https://github.com/CFCC/TurboPuppet}{https://github.com/CFCC/TurboPuppet}
\begin{outerlist}
    \item[]
            \begin{innerlist}
                \item Implemented Puppet to fully provision camper computers for a two-week summer program.
		 \item Built using many industry best practices learned at Constant Contact and through the community.
            \end{innerlist}
\end{outerlist}
\halfblankline

\textbf{Bluetooth Headphone Connection Manager} - \href{https://github.com/cohoe/maxime}{https://github.com/cohoe/maxime}
\begin{outerlist}
    \item[]
            \begin{innerlist}
                \item Tool in Python on Fedora Linux to manage PulseAudio sinks \& deal with Bluetooth connectivity quirks.
                \item Listens for DBus events and routes audio stream based on user configuration and connection state.
            \end{innerlist}
\end{outerlist}
\halfblankline

\textbf{Open Source Network Resource \& Access Manager} - \href{https://legacy-blog.grantcohoe.com/projects/starrs}{https://legacy-blog.grantcohoe.com/projects/starrs}
\begin{outerlist}
    \item[]
            \begin{innerlist}
                \item Designed and developed an interactive system for auditing and management of network resources.
                \item Gives users a self-service portal to provision logical resources such as IP addresses and DNS records.
            \end{innerlist}
\end{outerlist}
%% END PROJECTS %%


%% BEGIN EDUCATION %%
\section{Education}

\href{http://www.rit.edu/}{\textbf{Rochester Institute of Technology}},
Rochester, NY
\begin{outerlist}
	\item[] B.S., \href{http://www.nssa.rit.edu/}{Applied Networking \& System Administration}, May 2013

        Minors in \href{http://www.rit.edu/programs/minor_conc/database-design-and-development}
	{Database Design \& Development} and  \href{http://www.rit.edu/programs/minor_conc/applied-communication}
        {Applied Communication}
\end{outerlist}
%% END EDUCATION %%


%% BEGIN AWARDS %%
\section{Awards}

\textbf{Constant Contact Rovie's}
\begin{outerlist}
	\item[] \textit{Problem Solver Extraordinaire}
	\hfill \textbf{August 2019}

	\item[] \textit{Stabilizing Cassandra}
        \hfill \textbf{February 2019}
\end{outerlist}

\blankline

\textbf{Excellence@EMC Silver}
\begin{outerlist}
	\item[] \textit{Quick Connectivity Solution}
	\hfill \textbf{March 2016}

	\item[] \textit{Whatever It Takes}
	\hfill \textbf{January 2016}

	\item[] \textit{Transparent Proxy Implementation}
        \hfill \textbf{September 2014}
\end{outerlist}
%% END AWARDS %%


%% BEGIN ACTIVITIES/INTERESTS %%
\section{Activities \& Interests}
\begin{innerlist}
	\item Amateur ice hockey player in several Boston-area rec leagues.
	\item Home bartender and cocktail researcher.
	\item Frequent volunteer at a year-round YMCA summer camp on the shore of Lake Erie.
	\item Regular attendee of DevOps Days Boston and other local/national conferences and events.
	\item Skilled amateur tradesman in construction, carpentry, electrical, plumbing, and HVAC.
\end{innerlist}
%% END ACTIVITIES/INTERESTS %%


\end{document}
%%%%%%%%%%%%%%%%%%%%%%%%%% End CV Document %%%%%%%%%%%%%%%%%%%%%%%%%%%%%

%----------------------------------------------------------------------%
% The following is copyright and licensing information for
% redistribution of this LaTeX source code; it also includes a liability
% statement. If this source code is not being redistributed to others,
% it may be omitted. It has no effect on the function of the above code.
%----------------------------------------------------------------------%
% Copyright (c) 2007, 2008, 2009, 2010, 2011 by Theodore P. Pavlic
%
% Unless otherwise expressly stated, this work is licensed under the
% Creative Commons Attribution-Noncommercial 3.0 United States License. To
% view a copy of this license, visit
% http://creativecommons.org/licenses/by-nc/3.0/us/ or send a letter to
% Creative Commons, 171 Second Street, Suite 300, San Francisco,
% California, 94105, USA.
%
% THE SOFTWARE IS PROVIDED "AS IS", WITHOUT WARRANTY OF ANY KIND, EXPRESS
% OR IMPLIED, INCLUDING BUT NOT LIMITED TO THE WARRANTIES OF
% MERCHANTABILITY, FITNESS FOR A PARTICULAR PURPOSE AND NONINFRINGEMENT.
% IN NO EVENT SHALL THE AUTHORS OR COPYRIGHT HOLDERS BE LIABLE FOR ANY
% CLAIM, DAMAGES OR OTHER LIABILITY, WHETHER IN AN ACTION OF CONTRACT,
% TORT OR OTHERWISE, ARISING FROM, OUT OF OR IN CONNECTION WITH THE
% SOFTWARE OR THE USE OR OTHER DEALINGS IN THE SOFTWARE.
%----------------------------------------------------------------------%
